\documentclass[a4paper,12pt]{article}
\usepackage[T2A]{fontenc}
\usepackage[utf8]{inputenc}
\usepackage[russian, english]{babel}
\usepackage{indentfirst}
\usepackage{titlesec}
\usepackage{amsthm}
\usepackage{lipsum}

\title{Общая надстрока наименьшей длины}
\author{Андрей Осипов}

\newtheorem{theorem}{Теорема}
\newtheorem{lemma}{Лемма}
\newtheorem{definition}{Определение}

%%%%%%%%%% Document %%%%%%%%%%
\begin{document}
\Russian
\maketitle

\section{Постановка задачи}
Дан набор строк $S = \{s_1,\dots,s_n\}$ над конечным алфавитом, константного размера. 
Требуется найти, строку $s$ минимальной длины, содержащую как подстроку каждую строку из данного набора.
% 
%Требуется определить, существует ли строка $s$ длины не превышающей $k$, содержащая как подстроку каждую строку из данного набора.

Пусть теперь язык $L$ это множество пар вида $(S, k)$ для которых верно, что такая строка $s$ существует, и имеет длину не больше $k$.
Тогда можно говорить о том, что в таком виде задача разрешения языка $L$ является NP-полной.
Доказательство этого факта будет приведено ниже.
А пока мы ослабим условие следующим образом: пускай нам теперь нужно найти такую строку $t$, 
что она так же как и $s$ содержит всякую строку из $S$ как подстроку, и при этом $|t| \leq 4*|s|$

\section{NP-полнота}

\section{Алгоритм}

\section{Доказательство}

\begin{theorem}
the1
\end{theorem}

\end{document}
